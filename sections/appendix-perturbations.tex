\chapter{Left and Middle Perturbations}

\section{Identities}
\begin{equation}
\label{eq:identity_dxi}
\MatExp{( \TransformVector + \delta \TransformVector)} \approx \MatExp{(\LeftJacobianSE \delta \TransformVector)} \MatExp{\TransformVector},
\end{equation}

\begin{align}
\label{eq:identity_xi_small}
\Matlog{\Transform_1 \Transform_2} &= \Matlog{ \MatExp{\TransformVector_1} \MatExp{\TransformVector_2} } \nonumber \\
								  &\approx \left\{
	\begin{array}{ll}
		\Inv{\LeftJacobianSE(\TransformVector_2)} \TransformVector_1 + \TransformVector_2   & \mbox{if } \PredictionVector_1 ~ \mbox{small} \\
		\TransformVector_1 + \Inv{\LeftJacobianSE(-\TransformVector_1)} \TransformVector_2 & \mbox{if } \PredictionVector_2 ~ \mbox{small}.
	\end{array}
\right.
\end{align}

\section{Perturbing $\LieGroupSE{3}$}
Consider the quantity,
\begin{equation}
	\Transform_{ba} = \Transform_{bi} \Inv{\Transform_{ai}}
\end{equation}
where $\CoordinateFrame{i}$ is some inertial frame.


\subsection{Left Perturbation}
Separating $\Transform_{ba}$ into a mean component, $\Mean{\Transform}_{ba}$, and a small left perturbation, 
\begin{equation}
	\Transform_{ba} = \MatExp{\delta \TransformVector_{ba}^l} \Mean{\Transform}_{ba} = \MatExp{\delta \TransformVector_{ba}^l} \MatExp{\Mean{\TransformVector}_{ba}}
	\label{eq:left_perturb}
\end{equation}
Applying a logarithm to both sides,

\begin{equation}
	\Matlog{\Transform_{ba}} = \Matlog{\MatExp{\delta \TransformVector_{ba}^l} \MatExp{\Mean{\TransformVector}_{ba}}}
\end{equation}

Using \Cref{eq:identity_xi_small},

\begin{equation}
	\Matlog{\Transform_{ba}} \approx \Mean{\TransformVector}_{ba}
 + \Inv{\LeftJacobianSE}_{ba} \delta \TransformVector_{ba}^l
 \label{eq:left_perturb}
	\end{equation}
where $\LeftJacobianSE_{ba} \definedtobe \LeftJacobianSE(\Mean{\TransformVector}_{ba})$. This is exactly Equation 6.104 in Sean Anderson's thesis.



\subsection{Middle Perturbation}
Now consider the middle perturbation,
\begin{equation}
	\Transform_{ba} = \MatExp{\Mean{\TransformVector}_{ba} + \delta \TransformVector_{ba}^m}
\end{equation}
Immediately, we can take the logarithm of both sides and see that,

\begin{equation}
	\Matlog{\Transform_{ba}} = \Mean{\TransformVector}_{ba} + \delta \TransformVector_{ba}^m,
	\label{eq:middle_perturb}
\end{equation}
where we now observe that $\delta \TransformVector_{ba}^l \approx \LeftJacobianSE_{ba} \delta \TransformVector_{ba}^m$. 


\section{DPC $\LieGroupSE{3}$ Loss}
Using the notation in this document, the DPC derivation requires an expression for $\delta  \TransformVector_{b}$, and assumes that $\TransformVector_{a}$ is constant. 

\subsection{Middle Perturbation}


Consider,

\begin{equation}
	\Transform_{ba} = \Transform_{b} \Inv{\Transform_{a}}
\end{equation}
where we have dropped the $i$ frame for clarity. Middle perturbing $\Transform_{ba}$ and $\Transform_{b}$ and keeping $\Transform_{a}$ fixed (i.e., $\Transform_{a} = \Mean{\Transform}_{a}$).

\begin{equation}
	\MatExp{\Mean{\TransformVector}_{ba} + \delta \TransformVector_{ba}^m} = \MatExp{\Mean{\TransformVector}_{b} + \delta \TransformVector_{b}^m} \Inv{\Mean{\Transform}_{a}}
\end{equation}
Using \Cref{eq:identity_dxi} twice,

\begin{equation}
	\MatExp{(\LeftJacobianSE_{ba} \delta \TransformVector_{ba}^m)} \MatExp{\Mean{\TransformVector}_{ba}} = \MatExp{(\LeftJacobianSE_{b} \delta \TransformVector_{b}^m)} \MatExp{\Mean{\TransformVector}_{b}} \Inv{\Mean{\Transform}_{a}}
\end{equation}
Collecting terms, we have

\begin{equation}
	\MatExp{(\LeftJacobianSE_{ba} \delta \TransformVector_{ba}^m)} \Mean{\Transform}_{ba} = \MatExp{(\LeftJacobianSE_{b} \delta \TransformVector_{b}^m)} \Mean{\Transform}_{b} \Inv{\Mean{\Transform}_{a}}
\end{equation}
Right multiplying by $\Inv{\Mean{\Transform}}_{ba}$, we are left with


\begin{equation}
	\MatExp{(\LeftJacobianSE_{ba} \delta \TransformVector_{ba}^m)}  = \MatExp{(\LeftJacobianSE_{b} \delta \TransformVector_{b}^m)} 
\end{equation}

or

\begin{equation}
	\LeftJacobianSE_{ba} \delta \TransformVector_{ba}^m  = \LeftJacobianSE_{b} \delta \TransformVector_{b}^m.
\end{equation}
Solving for $\delta \TransformVector_{ba}^m$,

\begin{equation}
	 \delta \TransformVector_{ba}^m  =  \Inv{\LeftJacobianSE}_{ba} \LeftJacobianSE_{b} \delta \TransformVector_{b}^m.
	 \label{eq:xi_ba_in_terms_of_xi_b}
\end{equation}
Now inserting \Cref{eq:xi_ba_in_terms_of_xi_b} into \Cref{eq:middle_perturb},

\begin{equation}
	\Matlog{\Transform_{ba}} \approx \Mean{\TransformVector}_{ba} + \Inv{\LeftJacobianSE}_{ba} \LeftJacobianSE_{b} \delta \TransformVector_{b}^m
	\label{eq:lin_log_middle}
\end{equation}
This is exactly Equation 13 in the DPC-Net paper: 
\begin{equation}
\LogMapFunction{\PredictionVector + \delta \PredictionVector} \approx \Inv{\LeftJacobianSE(\LogMapFunction{\PredictionVector})} \LeftJacobianSE( \TransformVector) \delta \TransformVector + \LogMapFunction{\PredictionVector}.
\end{equation}

\subsection{Left Perturbation}

Using the left perturbation, we can repeat the procedure of relating $\delta \TransformVector_{ba}^l$ and $\delta \TransformVector_{b}^l$ (by perturbing $\Transform_{ba}$ and $\Transform_{b}$ and keeping $\Transform_{a}$ fixed (i.e., $\Transform_{a} = \Mean{\Transform}_{a}$).

\begin{equation}
	\MatExp{\delta \TransformVector_{ba}^l} \Mean{\Transform}_{ba} = \MatExp{\delta \TransformVector_{b}^l} \Mean{\Transform}_{b} \Inv{\Mean{\Transform}_{a}}
\end{equation}

\noindent from which we see immediately that $\MatExp{\delta \TransformVector_{ba}^l} = \MatExp{\delta \TransformVector_{b}^l}$ and therefore,

\begin{equation}
	\delta \TransformVector_{ba}^l = \delta \TransformVector_{b}^l
\end{equation}

\noindent Now using \Cref{eq:left_perturb}, we have

\begin{equation}
	\Matlog{\Transform_{ba}} \approx \Mean{\TransformVector}_{ba}
 + \Inv{\LeftJacobianSE}_{ba} \delta \TransformVector_{b}^l
\end{equation}

\subsection{Summary}

Using the left perturbation, we have

\begin{equation}
	\Matlog{\Transform_{ba}} \approx \Mean{\TransformVector}_{ba}
 + \Inv{\LeftJacobianSE}_{ba} \delta \TransformVector_{b}^l
\end{equation}

\noindent Using the centre/middle perturbation, we have

\begin{equation}
	\Matlog{\Transform_{ba}} \approx \Mean{\TransformVector}_{ba} + \Inv{\LeftJacobianSE}_{ba} \LeftJacobianSE_{b} \delta \TransformVector_{b}^m
\end{equation}

\noindent And we see the same earlier expression relating left and middle perturbations,

\begin{equation}
	\delta \TransformVector^l \approx \LeftJacobianSE \delta \TransformVector^m
\end{equation}

\subsection{Reconciliation}
Consider the two update rules:
\begin{equation}
\Transform_{b} \leftarrow \MatExp{\delta \TransformVector_b^l} \Mean{\Transform}_b	
\end{equation}
\begin{equation}
\Transform_{b} \leftarrow \MatExp{(\Mean{\TransformVector}_b + \delta \TransformVector_b^m)}
\end{equation}
But using \Cref{eq:identity_dxi}, the middle update becomes,
\begin{equation}
\MatExp{(\Mean{\TransformVector}_b + \delta \TransformVector_b^m)}
\approx \MatExp{( \LeftJacobianSE_{b} \delta \TransformVector_b^m)} \MatExp{\Mean{\TransformVector}_b} = \MatExp{( \LeftJacobianSE_{b} \delta \TransformVector_b^m)} \Mean{\Transform}_b = \MatExp{\delta \TransformVector_b^l} \Mean{\Transform}_b
\end{equation}
So the middle perturbation does not require us to keep the mean in the group (as long as we avoid any degeneracies).