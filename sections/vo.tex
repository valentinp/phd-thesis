\chapter{Classical Visual Odometry}
\label{ch:vo}
\epigraph{Eventually, my eyes were opened, and I really understood nature.}{\textsc{Claude Monet}}

\begin{figure}[h!]
\begin{center}
		\includegraphics[width=0.95\textwidth]{classical-vo/vo_taxonomy.pdf}
		\caption{A taxonomy of different types of visual odometry.}
  	\label{fig:vo_taxonomy}
\end{center}
\end{figure}


Visual odometry (VO) has a rich history in mobile robotics and computer vision. As this dissertation largely deals with the improvement of a baseline visual odometry pipeline, we first outline the components of what we have chosen to be a canonical VO system. For a seminal tutorial on visual odometry and its more general cousin, visual SLAM, we refer the reader to two seminal papers: \cite{Scaramuzza2011-qr} and \cite{Cadena2016-ds}.

\section{A taxonomy of VO}

VO can be largely divided along two dimensions (c.f. \Cref{fig:vo_taxonomy}): the type of camera (monocular vs. stereo) and the type of data association (indirect, or feature-based vs. direct, or pixel intensity-based). 

\textbf{Monocular vs. Stereo:}
The first distinction is based on the type of camera used by the VO pipeline. Monocular VO methods use a single camera to infer motion and can use a single compact, low-power vision sensor. They do not require any extrinsic calibration but must rely on known visual cues or external information (e.g., wheel odometry, inertial measurements) to provide metric egomotion estimates. Conversely, stereo VO methods use a stereo camera to triangulate objects with metric scale. This allows stereo VO to provide metricaly-acurate egomotion estimates. However, stereo methods rely on accurate extrinsic calibration, and their ability to resolve depth is limited by the baseline distance between the stereo pair. 

\textbf{Direct vs. Indirect:}
The second distinction is based on the type of data association used to match sequential images. Direct methods make the assumption of brightness constancy, and attempt to \textit{directly} maximize the similarity of pixel intensities. Indirect methods, however, rely on image features detectors to extract a set of salient landmarks, and then match these landmarks across images (typically through some sort of invariant descriptor).

\section{A classical VO pipeline}

In this thesis, we apply our learned pseudo-sensors to a baseline stereo, indirect visual odometry pipeline largely based on the work of \cite{furgale_phd11}. We choose this baseline system for its computational efficiency and robustness. We briefly summarize the main components of the dpipeline here.

\begin{figure}[h!]
\begin{center}
		\includegraphics[width=0.98\textwidth]{classical-vo/stereo_vo_pipeline.pdf}
		\caption{A `classical' stereo visual odometry pipeline consists of several distinct components that have interpretable inputs and outputs.}
  	\label{fig:vo_stereo_vo_pipeline}
\end{center}
\end{figure}

\subsection{Preprocessing}


\begin{figure}[h!]
     \centering
     \begin{subfigure}[b]{0.48\textwidth}
         \centering
     		\includegraphics[width=0.98\textwidth]{classical-vo/stereo_camera}
			\caption{Ideal stereo camera.}
			 \label{fig:vo_stereo_camera}
     \end{subfigure}
     \hfill
     \begin{subfigure}[b]{0.48\textwidth}
         \centering
         \includegraphics[width=0.75\textwidth]{classical-vo/stereo_rectification}
        \caption{Rectification and undistortion process. Figure adapted from \cite{florez2010}.}
         \label{fig:vo_undistort_recitfy}
	 \end{subfigure}
    \caption{Preprocessing components.}
        \label{fig:vo_preprocessing}
\end{figure}

During preprocessing, we use a lens model (assumed to be known apriori) to undistort each stereo image. Further, using the camera extrinsic parameters (also assumed to be known), we \textit{rectify} the stereo pair such that the images can be assumed to come from two cameras whose principal axes are parallel (\Cref{fig:vo_preprocessing}). Finally, we also assume that the stereo camera intrinsics are known a priori or compute them through a calibration process \citep{Furgale2013-sl}.

\subsection{Data Association}

\subsubsection{Feature Extraction and Matching}

\begin{figure}[h!]
\begin{center}
		\includegraphics[width=0.88\textwidth]{classical-vo/libviso2_scan}
		\caption{Feature tracking using \texttt{libviso2}, taken from \cite{Geiger2011-xe}. Colours correspond to depth.}
  	\label{fig:vo_feature_tracking}
\end{center}
\end{figure}

In this thesis, we focus on indirect stereo visual odometry for its computational efficiency. Although a number of different types of indirect feature extraction and matching methods can be used towards this end, we choose to use the \texttt{viso2} \citep{geiger2011stereoscan} image feature extraction and matching algorithm. In \texttt{viso2}, features are extracted using blob and corner masks with non-minimum and non-maximum suppression. Unlike other features detectors that do not assume a particular camera motion, \texttt{viso2} assumes a smooth camera trajectory that permits fast matching through a simple sum-of-absolute-difference error metric of $11 \times 11$ windows of Sobel filter responses. Finally, features are matched across a stereo-pair and forward in time, to ensure that a single feature exists across two consecutive stereo camera poses. 

Each extract feature corresponds to a
point in space, expressed in homogeneous coordinates in the camera frame as
$\HomogeneousPoint{i}{t} := \Transpose{\bbm p_1 & p_2 & p_3 & p_4 \ebm} \in
\HomogeneousNumbers[3]$.  Given our intrinsics and extrinsic calibration parameters, our idealized stereo-camera model, $\ProjectionFunction$,
projects a landmark expressed in homogeneous coordinates into image space, so
that $\ImageLandmark{i}{t}$, the stereo pixel coordinates of landmark $i$ in the first camera pose at time $t$, is given
by 
\begin{equation}
	\ImageLandmark{i}{t} = \bbm u_l \\ v_l \\ u_r \\ v_r \ebm 
  = \ProjectionFunction(\HomogeneousPoint{i}{t}) 
  = \Matrix{M} \frac{1}{p_3}\HomogeneousPoint{i}{t},
\end{equation}
where
\begin{equation}
 \Matrix{M} = \bbm f & 0 & c_u & f \frac{b}{2} \\ 0 & f & c_v & 0 \\ f 
                        & 0 & c_u & -f\frac{b}{2} \\ 0 & f & c_v & 0 \ebm.
\end{equation}
Here, $\{c_u, c_v\}$, $\{f_u, f_v\}$, and $b$ are the principal points, focal
lengths and baseline of the stereo camera respectively. Note that in this
formulation, the stereo camera frame is centered between the two individual
lenses. 


\subsubsection{Outlier Rejection}

To filter out any residual outlier matches, we use a three-point random sample consensus algorithm (RANSAC, \cite{FischlerRANSAC:1981}) based on an analytic solution to the six degree-of-freedom motion \citep{Umeyama1991-ws}.
  
\subsection{Maximum Likelihood Motion Solution}
We will define $\Transform_t \in\text{SE}(3)$, as the rigid transform between two subsequent stereo camera poses, 
$\CoordinateFrame{c_0}$ and  $\CoordinateFrame{c_1}$

\begin{equation}
	\Transform_t = \Transform_{c_1w} \Transform^{-1}_{c_0w},
\end{equation}
where $\CoordinateFrame{w}$ is a privileged world frame. After data association, we assume we have a set
of $N_t$ matches, $\{\ImageLandmark{i}{c_0}, \ImageLandmark{i}{c_1}\}_{i=1}^{N_t}$, between visual landmarks in the subsequent camera frames. For each match, we define an error function, $\Vector{e}_{i}(\Transform_t)$, that relates the rigid transform to these stereo feature matches. Throughout this dissertation, we assume that these errors are corrupted by zero-mean independent Gaussian noise with the (potentially heteroscedastic) covariance, $\Matrix{\Sigma}_{i,t}$;
 \begin{equation}
  \Vector{e}_{i}(\Transform_t) \sim
 \NormalDistribution\left(\Vector 0, \GeneralCovariance_{i,t}\right). 
\end{equation}
Under this noise model, the maximum likelihood transform, $\Transform^*_t$, is given by 
\begin{equation}
\label{eq:vo_objective}
  \Transform_t^* = \ArgMax{\Transform\in\text{SE}(3)}\prod_{i=1}^{N_t} p(\Vector{e}_{i}(\Transform_t)) = \ArgMin{\Transform\in\text{SE}(3)}\sum_{i=1}^{N_t} 
  \Transpose{\Vector{e}_{i}(\Transform_t)} \GeneralCovariance_{i,t}^{-1} \Vector{e}_{i}(\Transform_t).
\end{equation}
We will define the error function in two different ways.

\subsubsection{Point Cloud Error}
\label{sec:vo_point_cloud}
First, we can follow classical approach \citep{Maimone2007-tc} and define $ \Vector{e}_{i}(\Transform_t) $ based on a three-dimensional point cloud error. To do this, we invert our stereo camera model to triangulate pairs of points in each frame, $\HomogeneousPoint{i}{c_0} = \ProjectionFunction^{-1}(\ImageLandmark{i}{c_0})$ and $\HomogeneousPoint{i}{c_1} = \ProjectionFunction^{-1}(\ImageLandmark{i}{c_1})$,
\begin{align}
	 \Vector{e}_{i}(\Transform_t) = \Matrix{D}(\HomogeneousPoint{i}{c_1} - \Transform_t \HomogeneousPoint{i}{c_0}),
\end{align}
where $\Matrix{D} = \bbm \IdentityMatrix_{3\times3} & \Vector{0} \ebm \in \Real{}^{3\times4}$ converts homogenous coordinates into Euclidian coordinates.

We follow \cite{Maimone2007-tc} and assume each stereo projection is corrupted by additive Gaussian noise,

\begin{equation}
\ImageLandmark{i}{c} \sim \NormalDistribution\left(\bar{\Vector{y}}_{i,c}, \ImageLandmarkCovariance{i}{c} \right), 
\end{equation}  

then we can compute a density on the error function itself through first order noise propagation as
 \begin{equation}
  \Vector{e}_{i}(\Transform_t) \sim
 \NormalDistribution\left(\Vector 0, \GeneralCovariance_{i,t} \right), 
\end{equation}
where
\begin{equation}
	\GeneralCovariance_{i,t} = \Matrix{D}\Matrix{G}_{i,c_1} \ImageLandmarkCovariance{i}{c_1}  \Matrix{G}_{i,c_1}^T \Matrix{D}^T + 
 \Matrix{D}\Transform_t \Matrix{G}_{i,c_0}  \ImageLandmarkCovariance{i}{c_0} \Matrix{G}_{i,c_0}^T  \Transform_t^T \Matrix{D}^T
\end{equation}
with
$\Matrix{G}_{i,c} = \at{\PartialDerivative{\ProjectionFunction^{-1}}{\Vector{y}}}{\ImageLandmark{i}{c}}$.

\subsubsection{Reprojection Error}
\label{sec:vo_reprojection}

Alternatively, we can represent reprojection errors in the second frame directly as
\begin{equation}
  \Vector{e}_{i}(\Transform_t)  = \ImageLandmark{i}{c_1} - \ProjectionFunction( \Transform_t 
    \ProjectionFunction^{-1}( \ImageLandmark{i}{c_0} ) ),
   \label{eq:image_error}
\end{equation}
\noindent and model reprojection errors directly as 
 \begin{equation}
  \Vector{e}_{i}(\Transform_t) \sim \NormalDistribution\left(\Vector 0, \GeneralCovariance_{i,t} \right) = 
 \NormalDistribution\left(\Vector 0,  \ImageLandmarkCovariance{i}{t}\right), 
\end{equation}
where we abuse notation (slightly) and replace the index for the camera frames $c_0$ or $c_1$ with $t$ to indicate that this covariance refers to the reprojection error that involves both sets of features.

\subsubsection{Solution via Gauss-Newton Optimization}


 In either case, we have now defined a weighted nonlinear least squares problem which can be solved iteratively using
standard techniques. For our purposes, we opt to use Gauss-Newton optimization and follow \cite{Barfoot2017-ri} to optimize constrained poses.

Namely, at a given iteration $n$, we linearize the error function $\Vector{e}_{i}(\Transform_t)$, about an operating point $\Transform_t^{(n)} \in \LieGroupSE{3}$, which results in a quadratic approximation to \Cref{eq:vo_objective}. To linearize, we consider the left perturbations $\delta\Vector{\xi}\in\RealNumbers[6]$ represented in exponential
coordinates:
\begin{equation}
  \Transform_t = \MatExp{\delta\Vector{\xi}} \Transform_t^{(n)} \approx (\IdentityMatrix + \delta\Vector{\xi}^{\wedge}) \Transform_t^{(n)}.
\end{equation}
This allows us to transform \Cref{eq:vo_objective} into a linear least squares objective in $\delta\Vector{\xi}$:
\begin{equation}
\label{eq:vo_linleastsquares}
  \mathcal{L}(\delta \Vector{\xi}) = \frac{1}{2}\sum_{i=1}^{N_t} 
  \Transpose{\left(\Vector{e}_{i}
  - \Matrix J_{i} \delta\Vector{\xi}\right)}
\GeneralCovariance_{i}^{-1}
 \left(\Vector{e}_{i}
 - \Matrix J_{i} \delta\Vector{\xi}\right)
  \end{equation}
where $\Matrix J_{i} = \at{\PartialDerivative{\Vector{e}_i}{\delta \Vector{\xi}}}{\Transform_t^{(n)}}$, $\Vector{e}_{i} = \Vector{e}_{i}(\Transform_t^{(n)})$, and $\GeneralCovariance_{i} =  \GeneralCovariance_{i,t}(\Transform_t^{(n)})$. 
The minimum to this objective can be solved for analytically by solving the normal equations. This results in the optimal parameters,  
\begin{equation}
  \delta\Vector{\xi}^{\star} = 
  \left( \sum_{i=1}^{N_t} \Transpose{\Matrix J}_{i}
  \GeneralCovariance_{i}^{-1} \Matrix J_{i} \right)^{-1}
  \sum_{i=1}^{N_t} \Transpose{\Matrix J}_{i}
  \GeneralCovariance_{i}^{-1} \Vector{e}_{i}. 
\label{eq:least-squares-iteration}
\end{equation}
We then update the operating point and proceed to the next iteration,
\begin{equation}
  \Transform^{(n+1)} = \MatExp{\delta\Vector{\xi}^{\star}} \Transform^{(n)}. \label{eq:update}
\end{equation}
There are many reasonable choices for both the initial transform
$\Transform^{(0)}$ and for the conditions under which we terminate
iteration. We initialize the estimated transform to identity, and iteratively
perform the update given by \cref{eq:update} until we see a relative change in
the squared error of less than one percent after an update. 

%\newpage
%\section{Pose Graph Optimization}
%
%\begin{wrapfigure}{r}{0.4\textwidth}
%  \vspace{-20pt}
%  \begin{center}
%	\includegraphics[width=0.38\textwidth]{classical-vo/pose_graph.pdf}
%  \end{center}
%    \vspace{-20pt}
%	\label{fig:math_pose_graph}
%	\caption{The formulation of pose graph relaxation can incorporate different probabilistic \textit{factors} that constrain each camera pose.}
%\end{wrapfigure} 
%
%
%we fused our probabilistic rotation regression with classical stereo visual odometry using pose graph relaxation implemented with the help of a Python-based factor graph library which we will publicize after the review process. Using that framework, we solved
%\begin{align}
%	\Transform_{1,w}^*, \Transform_{2,w}^* &= \ArgMin{\Transform_{1,w}, \Transform_{2,w}\in\text{SE}(3)}\mathcal{L}(\Estimate{\Transform}_{2,1}, \Estimate{\Rotation}_{2,1}) \\ & = \ArgMin{\Transform_{1,w}, \Transform_{2,w}\in\text{SE}(3)} \Vector{\xi}_\text{1,2}^T \Matrix{\Sigma}^{-1}_\text{vo} \Vector{\xi}_\text{1,2} + \Vector{\phi}_\text{1,2}^T \Matrix{\Sigma}^{-1}_{\text{hn}} \Vector{\phi}_\text{1,2} 
%\end{align}
%where
%\begin{equation}
%	\Vector{\xi}_\text{1,2} =  \MatLog{\left(\Transform_{2,w} \Transform_{1,w}^{-1} \right)\Estimate{\Transform}_{2,1}^{-1}},
%\end{equation}
%\begin{equation}
%	\Vector{\phi}_\text{1,2} =  \MatLog{\left(\Rotation_{2,w} \Rotation_{1,w}^{T} \right)\Estimate{\Rotation}_{2,1}^{T}},
%\end{equation}
%and $\Estimate{\Transform}_{2,1}$, $\Matrix{\Sigma}_\text{vo}$ and $\Estimate{\Rotation}_{2,1}$, $\Matrix{\Sigma}_{\text{hn}}$ were provided by our classical estimator and the HydraNet network respectively. Note that $\Matrix{\Sigma}_{\text{hn}} \in \Real^{3 \times 3} \geq 0$ while $\Matrix{\Sigma}_{\text{vo}} \in \Real^{6 \times 6} \geq 0 $. We also overload the logarithm function, $\MatLog{\cdot}$ to represent both $\LieGroupSE{3}$ and $\LieGroupSO{3}$ logarithmic maps as necessary. To account for gauge freedom, we fixed the first transformation to identity, $\Transform_{1,w} = \IdentityMatrix$, and initialized $\Transform_{2,w}$ to  $\Estimate{\Transform}_{2,1}$.  After convergence, we composed the final frame-to-frame estimate as $ \Transform_{2,1}^* =  \Transform_{2,w}^*  \left(\Transform_{1,w}^*\right)^{-1} = \Transform_{2,w}^*$.
%



\section{Robust Estimation}
Since \cref{eq:vo_linleastsquares} assigns cost values that grow quadratically with measurement error, it is very sensitive to outlier measurements.
A common solution to this problem is to replace the $L_2$ cost function with one that is less sensitive to large measurement errors \citep{MacTavish2015-wt}.
These robust cost functions are collectively known as M-estimators, and many variants exist. Each uses a re-weighting function, $\rho(\cdot)$,

\begin{equation}
  \Transform^* = \ArgMin{\Transform\in\text{SE}(3)}\sum_{i=1}^{N_t} 
  \rho\left(\Transpose{\Vector{e}_{i}} \GeneralCovariance_{i,t}^{-1} \Vector{e}_{i}\right) = \ArgMin{\Transform\in\text{SE}(3)}\sum_{i=1}^{N_t} 
  \rho(\epsilon_i),
\end{equation}

where. given a parameter $c$, some common examples include:

\begin{align}
\rho(\epsilon) &= \left\{  	\begin{array}{ll}
		 \frac{c^2}{2} \log{\left(1 + \frac{\epsilon^2}{c^2}\right)}   & \mbox{Cauchy,} \\
		 \frac{1}{2} \frac{\epsilon^2}{c^2 + \epsilon^2}  & \mbox{Geman-McClure \citep{geman1992nonlinear},} \\
		 \\
		\left\{  	\begin{array}{ll}  \frac{\epsilon^2}{2} & \mbox{if} \Norm{\epsilon} < c \\
										c\Norm{\epsilon} - \frac{c^2}{2} & \mbox{if} \Norm{\epsilon} \geq c \end{array}
																						 \right.   & \mbox{Huber \citep{huber1964robust}.} \\
	\end{array}
	\right.
\end{align}




\section{Outstanding Issues}
There are several outstanding limitations of classical visual odometry pipelines that we can address with learned pseudo-sensors. 

\begin{table}[h!]
	\caption{\textbf{Data efficiency vs. computational efficiency}}	\begin{threeparttable}
	\begin{tabular}{m{0.68\textwidth}m{0.28\textwidth}}
		\toprule
		\textbf{Synopsis} & \textbf{Addressed by} \\ \midrule  
		Classical VO pipelines face a difficult-to-optimize trade-off between using all of the information contained within image and while still remaining computationally tractable.  & PROBE, DPC-Net, Sun-BCNN, HydraNet \\
		& \\
		\bottomrule
	\end{tabular}
\end{threeparttable}
\end{table}


\begin{table}[h!]
	\caption{\textbf{Systematic bias}}
	\begin{threeparttable}
	\begin{tabular}{m{0.68\textwidth}m{0.28\textwidth}}
		\toprule
		\textbf{Synopsis} & \textbf{Addressed by} \\ \midrule  
		Stereo visual odometry can incur systematic bias through poor extrinsic or intrinsic calibration, stereo triangulation errors, poor feature \textit{spread} (i.e., concentration of features on one side of an image), and poor data association due self-similar textures. &  DPC-Net \\
		& \\
		\bottomrule
	\end{tabular}
\end{threeparttable}
\end{table}


\begin{table}[h!]
	\caption{\textbf{Homoscedastic uncertainty}}
	\begin{threeparttable}
	\begin{tabular}{m{0.68\textwidth}m{0.28\textwidth}}
		\toprule
		\textbf{Synopsis} & \textbf{Addressed by} \\ \midrule  
		Stationary, homoscedastic noise in observation models can often reduce the consistency and accuracy of state estimates. This is especially true for complex, inferred measurement models. In visual data, inferred visual observations can be degraded not only due to sensor imperfections (e.g. poor intrinsic calibration, digitization effects, motion blur), but also as a result of the observed environment (e.g. self-similar scenes, specular surfaces, textureless environments). &  PROBE, Sun-BCNN, HydraNet \\
		& \\
		\bottomrule
	\end{tabular}
\end{threeparttable}
\end{table}

%
%
%\textbf{Adaptation to new environments}
%
%\begin{table}[h!]
%	\begin{threeparttable}
%	\begin{tabular}{m{0.68\textwidth}m{0.28\textwidth}}
%		\toprule
%		\textbf{Synopsis} & \textbf{Addressed by} \\ \midrule  
%		Traditional VO pipelines do little to adapt to new environments.
%. &  PROBE-GK, DPC-Net, HydraNet \\
%		& \\
%		\bottomrule
%	\end{tabular}
%\end{threeparttable}
%\end{table}

